\documentclass{beamer}
\usepackage[ngerman]{babel}
\usepackage[utf8]{inputenc}
\usepackage{graphicx}
\usepackage{hyperref}
%\usepackage[plainpages=false, pdfpagelabels, colorlinks=true, linkcolor=black, menucolor=black, urlcolor=black, citecolor=black, pdftitle={Entwicklung eines Frameworks fuer die Verteilungsoptimierung in Publish/Subscribe Systemen auf Basis eines strukturierten P2P-Overlay Netzwerks}, pdfauthor={Johannes Held}, pdfsubject={Abschlussvortrag}, pdfkeywords={event dissemination, p2p-overlay network, publish/subscribe system, decentralized}]{hyperref}

\usepackage{multicol}
\usepackage{multirow}
\usepackage{booktabs}
\usepackage[babel,german=quotes]{csquotes}

\usecolortheme{crane}

\title{Entwicklung eines Frameworks für die Verteilungsoptimierung in Publish/Subscribe-Systemen auf Basis eines strukturierten P2P-Overlay-Netzwerks}
\author{Johannes Held\\\url{johannes.held@cs.fau.de}}
\date{17.\,01.\,2011}

\begin{document}


\frame{\titlepage}

\frame{\tableofcontents}

\section{Einleitung}
\frame {
	\frametitle{Einleitung}
}
\section{Szenario}
\frame {
	\frametitle{Szenario}
}
\section{Klassifizierung von Eventtypen}
\frame {
	\frametitle{Klassifizierung von Eventtypen}
}
\section{Strukturierte p2p-Netzwerke}
\frame {
	\frametitle{Strukturierte p2p-Netzwerke}
}
\subsection{KBR-API}
\frame {
	\frametitle{KBR-API}
}
\section{Umsetzung der Dimensionen}
\frame {
	\frametitle{Umsetzung der Dimensionen}
}
\section{Konzeption des Publish/Subscribe-Systems}
\frame {
	\frametitle{Konzeption des Publish/Subscribe-Systems}
}

\section{Fragen?}
\frame{
	\frametitle{Fragen?}
	Gerne!\\
	$\dots$ vielen Dank.
}


%	\begin{block}{O-Ton Linus Torvalds}
%	DVCS are cool.\\
%	\end{block}

\end{document}
