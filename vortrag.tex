\documentclass[professionalfonts]{beamer}
\usepackage[utf8]{inputenc}
\usepackage[ngerman]{babel}
\usepackage{graphicx}
\usepackage{hyperref}
%\usepackage[plainpages=false, pdfpagelabels, colorlinks=true, linkcolor=black, menucolor=black, urlcolor=black, citecolor=black, pdftitle={Entwicklung eines Frameworks fuer die Verteilungsoptimierung in Publish/Subscribe Systemen auf Basis eines strukturierten P2P-Overlay Netzwerks}, pdfauthor={Johannes Held}, pdfsubject={Abschlussvortrag}, pdfkeywords={event dissemination, p2p-overlay network, publish/subscribe system, decentralized}]{hyperref}

\usepackage{multicol}
\usepackage{multirow}
\usepackage{booktabs}
\usepackage[babel,german=quotes]{csquotes}

\usecolortheme{crane}

\title{Entwicklung eines Frameworks für die Verteilungsoptimierung in Publish/Subscribe-Systemen auf Basis eines strukturierten P2P-Overlay-Netzwerks}
\author{Johannes Held\\\url{johannes.held@cs.fau.de}}
\date{17.\,01.\,2011}

\begin{document}


\frame{\titlepage}

\frame{\tableofcontents}

\section{Einleitung}
\frame {
	\frametitle{Einleitung}
	Ich will erzählen warum. Warum p2p, warum verteilt?
	
	\begin{block}{Kommunikation via Broadcast}
		\begin{itemize}
			\item $n$ Knoten; je 10 Bytes pro Update
			\item Framerate 30 Hz (30 Updates/s)
			\item pro Frame 0,3 KB/s an Server
			\item Server hat 100 Mbit/s (12500 KB/s)
			\item $\lfloor\sqrt{12500\,KB/s \div 0,3\,KB/s}\rfloor = 204$
		\end{itemize}
	 \end{block}
}

\frame {
	\frametitle{M$^2$etis}
	\begin{figure}[htbp]
	\centering
	\resizebox{\textwidth}{!}{%
	\includegraphics{grafics/metis_aufbau.pdf}}
	\caption{Architekturübersicht von M$^2$etis}
	\label{fig:metis_aufbau}
	\end{figure}
}

\frame {
	\frametitle{M$^2$etis}
	\begin{figure}[htbp]
	\resizebox{0.5\textwidth}{!}{%
	\includegraphics{grafics/metis_aufbau.pdf}}
	\label{fig:metis_aufbau}
	\end{figure}
	\begin{enumerate}
	\item Semantische Beschreibung der Eventtypen
	\item Generierung der optimierten Kanäle
	\item Benutzung des PubSubSytems
	\end{enumerate}
}



\section{Szenario}
\frame {
	\frametitle{Szenario}
}
\section{Klassifizierung von Eventtypen}
\frame {
	\frametitle{Klassifizierung von Eventtypen}
}
\section{Strukturierte p2p-Netzwerke}
\frame {
	\frametitle{Strukturierte p2p-Netzwerke}
	\begin{itemize}
		\item Einarbeiten und Verstehen von CAN, Chord und Pastry
		\item Suche 
	\end{itemize}
}
\subsection{KBR-API}
\frame {
	\frametitle{KBR-API}
}
\section{Umsetzung der Dimensionen}
\frame {
	\frametitle{Umsetzung der Dimensionen}
	\begin{description}
	\item[Verteilung] bestimmt die Verteilungsart der einzelnen Events und den Aufbau des logischen Multicast-Trees, mittels dessen die Nachrichten versandt werden \cite{KostasKatrinis2005}.
	\item[Filterung] erlaubt es Anmeldungen, Prädikate mitzugeben. Implementierungen dieser Policy müssen sicherstellen, dass diese Prädikate nach oben im Multicast-Tree zusammengeführt und Nachrichten frühzeitig gefiltert werden können. Dies bedeutet, dass Nachrichten jeweils beim Versand durch den logischen Kopf des Multicast-Trees gefiltert werden.
	\item[Zustellung] bestimmt das Kommunikationsparadigma des Nachrichtenversands und leitet beispielsweise den Versand von Bestätigungen über eingegangene Nachrichten an den sendenden Knoten ein.
	\item[Reihenfolge] definiert das Synchronisationskonzept eines Kanals.
	\item[Persistenz] bietet die Möglichkeit der Speicherung eines Events beziehungsweise der daraus erfolgenden Zustandsänderung der virtuellen Welt.
	\item[Sicherheit] gibt eine Schnittstelle zur Nachrichtenverschlüsselung vor.
	\item[Validität] prüft die ankommenden Nachrichten auf ihre Validität. Frühzeitig verworfene Nachrichten vermindern das Nachrichtenaufkommen im System stark.
	\end{description}
}
\section{Konzeption des Publish/Subscribe-Systems}
\frame {
	\frametitle{Konzeption des Publish/Subscribe-Systems}
}

\section{Hindernisse bei der Implementierung}
\frame{
	\frametitle{Hindernisse bei der Implementierung}
	\begin{block}{Knotenausfall wird nicht erkannt}
		\pause
		\dots obwohl Chimera-Code das Nachrichtensenden prüft
		\pause
		\begin{itemize}
			\item Chimera sendet Nachrichten mit UDP
			\item $\rightarrow$ Heartbeat-Mechanismus zur Erkennung
		\end{itemize}
	\end{block}
	\pause
	\begin{block}{Richtige Kommunikation schwer prüfbar}
		\pause
		\begin{itemize}
			\item Zu wenige Knoten $\rightarrow$ Fehlerhafte Einteilung in Routingtabellen
			\item Zu viele Knoten $\rightarrow$ Debug-Ausgaben nicht mehr erfassbar
		\end{itemize}		
	\end{block}

}


\section{Offene Punkte}
\frame{
	\frametitle{Offene Punkte}
	\begin{block}{Weiterführende Arbeiten}
	\begin{itemize}[<+->]
		\item Kostenmodell \& Semantische Beschreibung \& DSL

		\item Strategies, Strategies, Strategies
		\item M$^2$etis-Optimierer
		\item Simulator (bsp. OverSim)
	\end{itemize}
	\end{block}
}

\section{Fragen?}
\frame{
	\frametitle{Fragen?}
	Gerne!\\
	$\dots$ vielen Dank.
}


\end{document}
